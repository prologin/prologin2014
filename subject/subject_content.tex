\section{Introduction}

Ce mercredi soir, après avoir joué à des jeux de plateau avec les autres participants, le candidat\footnote{Celui qui lit présentement ce texte.} se rendit dans le dortoir. Sur son chemin, il tomba nez à nez avec une bestiole toute mignonne\footnote{\J /人{\A ◕‿‿◕}人\.} qui se mit à lui parler :

\noindent
\og Ça t'intéresserait d'avoir des pouvoirs magiques ? \fg

Le candidat s'arrêta. Était-ce vraiment cet animal qui lui avait parlé ? Ses lèvres n'avaient pourtant pas bougé. La bestiole continua.

\noindent
\og Oui, c'est bien à toi que je parle. Ça ne te dirait pas d'être capable en un clin d'œil de construire des tours de magie ?\\
--- Attends, par tour de magie, tu veux dire genre lapin, chapeau et tout ?\\
--- Je parle de tourelles.\\
--- Pour quoi faire ?\\
--- Pour tuer des sorciers.\\
--- Pour quoi faire ?\\
--- Pour pouvoir explorer plus librement la carte.\\
--- Pour quoi faire ?\\
--- Pour obtenir le PLUS GROS SCORE.\\
--- Ah, cool\footnote{Si le candidat était une candidate, elle ne réagirait probablement pas ainsi.} !\\
--- Si ça t'intéresse, tu as juste à faire un pacte avec moi\footnote{NON C'EST UN PIÈGE}. Si tu es mineur, tes parents ont déjà signé l'autorisation parentale.\\
--- Mais, mais, et les autres candidats, alors ?\\
--- Tu sais, tous les autres candidats ont effectivement conclu ce pacte. Il ne reste plus que toi.\\
--- Ah ben d'accord alors\footnote{Voyons. Ne sois pas candide.}. \fg

\subsection{L'initiation}

La bestiole emmena le candidat dans une pièce carrée, où trois candidats étaient déjà disposés dans les coins. Au centre, un superbe artefact brillait de mille feux. À mi-chemin de chacun des quatre murs, une fontaine magique.

\noindent
\og Pas terrible, la déco.\\
--- Je te trouve bien impertinent ! \fg{} s'écria la bestiole, sans s'arrêter de sourire.

\noindent
\og Choisis un endroit pour construire ta première tourelle.\\
--- Ici ? \fg{} demanda le candidat, en désignant une dalle de la salle. À cet instant précis, une tourelle s'éleva.

\noindent
\og Tu peux aussi la détruire, en prononçant la formule magique.\\
--- \og S'il te plaît \fg{} ?\\
--- Mais non. Tu ne connais pas d'incantation ?\\
--- Mais je ne connais que \og \textbf{Wingardium Leviosa}. \fg\\
--- Non, mais c'est presque ça. Qu'est-ce qui correspond à l'effondrement d'une architecture, selon toi ? \fg

Le candidat se rappela effectivement qu'un de ses souvenirs semblait correspondre à la description.

\noindent
\og \textbf{Windowsium Levista} ! \fg{} À ces mots, la tour s'effondra brusquement.

La voix continua.

\noindent
\og Tu peux également créer des sorciers depuis ta base, qui se déplacent sur la carte pour attaquer d'autres sorciers ainsi que des tourelles adverses.\\
--- Pour quoi faire ?\\
--- Pour pouvoir faire accéder de tes sorciers à un coin de la salle, ce qui anéantira le joueur adverse !\\
--- Pour quoi faire ?\\
--- Pour que le candidat soit mort.\\
--- Pour quoi faire ?\\
--- Pour avoir plus de rab' au banquet du samedi.\\
--- Ah, cool\footnote{Si le candidat était une candidate, elle réagirait probablement ainsi.} !\fg

\subsection{Dénouement}

La bestiole était claire sur ce point : les effets ne dureraient que trente-six heures.