\section{Introduction}

Ce mercredi soir, après avoir joué à des jeux de plateau avec les autres participants, le candidat\footnote{Celui qui lit présentement ce texte.} se rendit dans le dortoir. Sur son chemin, il tomba nez à nez avec une bestiole toute mignonne\footnote{\J /人{\A ◕‿‿◕}人\.} qui se mit à lui parler :

\noindent
\og Ça t'intéresserait d'avoir des pouvoirs magiques ? \fg

Le candidat s'arrêta. Était-ce vraiment cet animal qui lui avait parlé ? Ses lèvres n'avaient pourtant pas bougé. La bestiole continua.

\noindent
\og Oui, c'est bien à toi que je parle. Ça ne te dirait pas d'être capable en un clin d'œil de construire des tours de magie ? \fg

Le candidat chercha aussitôt un bouton \og \emph{Report Spam} \fg{} sur les murs mais, n'en trouvant pas, il se décida à répondre.

\noindent
\og Attends, par tour de magie, tu veux dire genre lapin, chapeau et tout ?\\
--- Je parle de tourelles.\\
--- Pour quoi faire ?\\
--- Pour tuer des sorciers.\\
--- Pour quoi faire ?\\
--- Pour pouvoir explorer plus librement la carte.\\
--- Pour quoi faire ?\\
--- Pour obtenir le PLUS GROS SCORE.\\
--- Ah, cool ! \fg

Profitant de l'émerveillement du candidat, le mystérieux animal ajouta :

\noindent
\og Vendredi soir, ce sera la \emph{nuit de Walpurgis}. Samedi à 00 h 42, le potentiel magique sera à son paroxysme\footnote{Le sommeil aussi, d'ailleurs.} et c'est pourquoi nous devons d'ici là déterminer les dix meilleurs magiciens, qui auront l'honneur de passer devant un grand conseil d'archimages à l'aube. \fg

La bestiole laissa passer un silence, puis reprit :

\og Tu as le pouvoir de changer le cours des choses. Tu peux devenir le meilleur développeur de France\footnote{Et gratuitement.}.\\
--- Même si je ne sais coder qu'en PHP ? \fg

La bestiole resta muette.

\noindent
\og … Et comment puis-je acquérir ces pouvoirs ?\\
--- Tu as juste à faire un pacte avec moi\footnote{NON C'EST UN PIÈGE}. Si tu es mineur, tes parents ont déjà signé l'autorisation parentale.\\
--- Et qu'en est-il des autres candidats, alors ?\\
--- Tu sais, tous les autres candidats ont effectivement conclu ce pacte avec moi, et ont accepté les conditions d'utilisation\footnote{Cf. \emph{Terms of Service; Didn't Read}, \url{http://tosdr.org}}. Il ne reste plus que toi.\\
--- Ah, ben d'accord alors\footnote{Voyons. Ne sois pas candide.}. \fg

\subsection{L'initiation}

La bestiole emmena le candidat dans une pièce carrée, dont trois des coins étaient déjà occupés par des candidats. Au centre, un superbe artefact brillait de mille feux. À mi-chemin de chacun des quatre murs, une fontaine magique.

\noindent
\og Pas terrible, la déco.\\
--- Je te trouve bien impertinent ! \fg{} s'écria la bestiole, sans s'arrêter de sourire. \og Tous ces objets que tu vois sont des gisements de magie. \fg

\noindent
\og Choisis un endroit pour construire ta première tourelle.\\
--- Ici ? \fg{} demanda le candidat, en désignant une dalle de la salle. À cet instant précis, une tourelle s'y éleva. Le candidat recula de surprise.

\noindent
\og Tu peux aussi la détruire, en prononçant la formule magique.\\
--- \og S'il te plaît \fg{} ?\\
--- Mais non. Tu ne connais pas d'incantation ?\\
--- Je ne connais que \og \textbf{Wingardium Leviosa}. \fg\\
--- Non, mais c'est presque ça. Qu'est-ce qui correspond à l'effondrement d'une architecture, selon toi ? \fg

Le candidat se rappela effectivement qu'un de ses souvenirs semblait correspondre à la description.

\noindent
\og \textbf{Windowsium Levista} ! \fg{} À ces mots, la tour s'effondra brusquement.

La voix continua.

\noindent
\og Détruire une tour te rapporte des points de magie. Tu peux également créer des sorciers depuis ta base, qui se déplacent sur la carte pour attaquer d'autres sorciers ainsi que des tourelles adverses, ou récolter des points de magie sur l'artefact ou les fontaines.\\
--- Pour quoi faire ?\\
--- Pour faire progresser tes sorciers jusqu'à un coin de la salle, ce qui anéantira le joueur adverse correspondant !\\
--- Pour quoi faire ?\\
--- Pour que le candidat soit mort.\\
--- Pour quoi faire ?\\
--- Pour avoir du rab' au banquet du samedi.\\
--- Ah, cool !\fg

\subsection{Dénouement}

\noindent
\og Dans les clauses du pacte, je peux exaucer ce que tu veux. Quel est ton vœu ?\\
--- J'aimerais avoir deux vœux.\\
--- Les explosions combinatoires ne sont pas acceptées. Que désires-tu le plus au monde ? \fg

Le candidat s'arrêta et réfléchit.

\noindent
\og … Un… un ordinateur portable.\\
--- Eh bien, combats tous les candidats, et tu l'obtiendras. Bonne chance ! \fg

La bestiole s'éclipsa.

\noindent
\og Attends ! Qui es-tu vraiment ?\\
--- Je suis… \fg

La bestiole se retourna et amplifia son sourire.

\noindent
\og … un incubateur. \fg