

%
% This file was generated using gen/make_tex.rtex
% Do not modify unless you are absolutely sure of what you are doing
%



\noindent \begin{tabular}{lp{15cm}}
\textbf{Constante:} & TAILLE\_TERRAIN \\
\textbf{Valeur:} & 31 \\
\textbf{Description:} & Taille du terrain (longueur et largeur) \\
\end{tabular} 
\vspace{0.2cm} \\



\noindent \begin{tabular}{lp{15cm}}
\textbf{Constante:} & NB\_JOUEURS \\
\textbf{Valeur:} & 4 \\
\textbf{Description:} & Nombre de joueurs dans la partie \\
\end{tabular} 
\vspace{0.2cm} \\



\noindent \begin{tabular}{lp{15cm}}
\textbf{Constante:} & MAX\_TOUR \\
\textbf{Valeur:} & 100 \\
\textbf{Description:} & Nombre maximum de tours à jouer avant la fin de la partie \\
\end{tabular} 
\vspace{0.2cm} \\



\noindent \begin{tabular}{lp{15cm}}
\textbf{Constante:} & MAGIE\_TOUR \\
\textbf{Valeur:} & 20 \\
\textbf{Description:} & Magie gagnée à chaque tour \\
\end{tabular} 
\vspace{0.2cm} \\



\noindent \begin{tabular}{lp{15cm}}
\textbf{Constante:} & MAGIE\_FONTAINES \\
\textbf{Valeur:} & 15 \\
\textbf{Description:} & Magie gagnée à chaque tour pour chaque fontaine possédée \\
\end{tabular} 
\vspace{0.2cm} \\



\noindent \begin{tabular}{lp{15cm}}
\textbf{Constante:} & MAGIE\_COMBAT \\
\textbf{Valeur:} & 1 \\
\textbf{Description:} & Magie gagnée à chaque sorcier tué \\
\end{tabular} 
\vspace{0.2cm} \\



\noindent \begin{tabular}{lp{15cm}}
\textbf{Constante:} & MAGIE\_SUPPRESSION \\
\textbf{Valeur:} & 10 \\
\textbf{Description:} & Magie récupérée à chaque tourelle supprimée \\
\end{tabular} 
\vspace{0.2cm} \\



\noindent \begin{tabular}{lp{15cm}}
\textbf{Constante:} & COUT\_SORCIER \\
\textbf{Valeur:} & 2 \\
\textbf{Description:} & Nombre de points de magie par sorcier \\
\end{tabular} 
\vspace{0.2cm} \\



\noindent \begin{tabular}{lp{15cm}}
\textbf{Constante:} & COUT\_TOURELLE \\
\textbf{Valeur:} & 20 \\
\textbf{Description:} & Nombre de points de magie par tourelle \\
\end{tabular} 
\vspace{0.2cm} \\



\noindent \begin{tabular}{lp{15cm}}
\textbf{Constante:} & COUT\_PORTEE \\
\textbf{Valeur:} & 4 \\
\textbf{Description:} & Coût de base pour chaque case de portée supplémentaire \\
\end{tabular} 
\vspace{0.2cm} \\



\noindent \begin{tabular}{lp{15cm}}
\textbf{Constante:} & PORTEE\_SORCIER \\
\textbf{Valeur:} & 4 \\
\textbf{Description:} & Nombre maximum de cases qu'un sorcier peut franchir à chaque tour \\
\end{tabular} 
\vspace{0.2cm} \\



\noindent \begin{tabular}{lp{15cm}}
\textbf{Constante:} & PORTEE\_TOURELLE \\
\textbf{Valeur:} & 3 \\
\textbf{Description:} & Portée de base d'une tourelle \\
\end{tabular} 
\vspace{0.2cm} \\



\noindent \begin{tabular}{lp{15cm}}
\textbf{Constante:} & CONSTRUCTION\_TOURELLE \\
\textbf{Valeur:} & 3 \\
\textbf{Description:} & Portée de construction des tourelles \\
\end{tabular} 
\vspace{0.2cm} \\



\noindent \begin{tabular}{lp{15cm}}
\textbf{Constante:} & VIE\_TOURELLE \\
\textbf{Valeur:} & 10 \\
\textbf{Description:} & Points de vie d'une tourelle à sa création \\
\end{tabular} 
\vspace{0.2cm} \\



\noindent \begin{tabular}{lp{15cm}}
\textbf{Constante:} & ATTAQUE\_TOURELLE \\
\textbf{Valeur:} & 10 \\
\textbf{Description:} & Points d'attaque d'une tourelle au début d'un tour \\
\end{tabular} 
\vspace{0.2cm} \\



\noindent \begin{tabular}{lp{15cm}}
\textbf{Constante:} & POINTS\_SURVIVRE \\
\textbf{Valeur:} & 1 \\
\textbf{Description:} & Points gagnés pour avoir survécu à la fin de la partie \\
\end{tabular} 
\vspace{0.2cm} \\



\noindent \begin{tabular}{lp{15cm}}
\textbf{Constante:} & POINTS\_VAINQUEUR \\
\textbf{Valeur:} & 1 \\
\textbf{Description:} & Points gagnés pour avoir vaincu un adversaire \\
\end{tabular} 
\vspace{0.2cm} \\



\noindent \begin{tabular}{lp{15cm}}
\textbf{Constante:} & POINTS\_CONTROLE\_FONTAINE \\
\textbf{Valeur:} & 1 \\
\textbf{Description:} & Points gagnés pour contrôler une fontaine à la fin de la partie \\
\end{tabular} 
\vspace{0.2cm} \\



\noindent \begin{tabular}{lp{15cm}}
\textbf{Constante:} & POINTS\_CONTROLE\_ARTEFACT \\
\textbf{Valeur:} & 4 \\
\textbf{Description:} & Points gagnés pour contrôler un artefact à la fin de la partie \\
\end{tabular} 
\vspace{0.2cm} \\





\functitle{case\_info} \\
\noindent
\begin{tabular}[t]{@{\extracolsep{0pt}}>{\bfseries}lp{10cm}}
Description~: & Information sur les cases \\
Valeurs~: &
\small
\begin{tabular}[t]{@{\extracolsep{0pt}}lp{7cm}}
    
        \textsl{CASE\_SIMPLE}~: & Case simple \\
    
        \textsl{CASE\_TOURELLE}~: & Tourelle \\
    
        \textsl{CASE\_BASE}~: & Base du joueur \\
    
        \textsl{CASE\_FONTAINE}~: & Fontaine magique \\
    
        \textsl{CASE\_ARTEFACT}~: & Artefact magique \\
    
        \textsl{CASE\_ERREUR}~: & Erreur \\
    
\end{tabular} \\
\end{tabular}



\functitle{erreur} \\
\noindent
\begin{tabular}[t]{@{\extracolsep{0pt}}>{\bfseries}lp{10cm}}
Description~: & Erreurs possibles \\
Valeurs~: &
\small
\begin{tabular}[t]{@{\extracolsep{0pt}}lp{7cm}}
    
        \textsl{OK}~: & L'action s'est effectuée avec succès \\
    
        \textsl{ANNULER\_IMPOSSIBLE}~: & Aucune action à annuler \\
    
        \textsl{CASE\_IMPOSSIBLE}~: & Cette case n'existe pas \\
    
        \textsl{CASE\_ADVERSE}~: & Vous ne contrôlez pas cette case \\
    
        \textsl{CASE\_UTILISEE}~: & Cette case n'est pas libre \\
    
        \textsl{CASE\_VIDE}~: & Cette case est vide \\
    
        \textsl{VALEUR\_INVALIDE}~: & Cette valeur est invalide \\
    
        \textsl{MAGIE\_INSUFFISANTE}~: & Vous n'avez pas assez de magie \\
    
        \textsl{SORCIERS\_INSUFFISANTS}~: & Vous n'avez pas assez de sorciers \\
    
        \textsl{ATTAQUE\_INSUFFISANTE}~: & Vous n'avez pas assez de points d'attaque \\
    
        \textsl{PHASE\_INCORRECTE}~: & Cette action ne peut pas être utilisée lors de cette phase du jeu. \\
    
        \textsl{PORTEE\_INSUFFISANTE}~: & Vous n'avez pas assez de portée pour effectuer cette action \\
    
        \textsl{PERDANT}~: & Vous avez perdu et ne pouvez pas effectuer d'actions \\
    
\end{tabular} \\
\end{tabular}





\functitle{position}

\begin{lst-c++}
struct position {
    int x;
    int y;
};
\end{lst-c++}

\noindent
\begin{tabular}[t]{@{\extracolsep{0pt}}>{\bfseries}lp{10cm}}
Description~: & Représente la position sur la carte \\
Champs~: &
\small
\begin{tabular}[t]{@{\extracolsep{0pt}}lp{7cm}}
    
        \textsl{x}~: & Coordonnée en X \\
    
        \textsl{y}~: & Coordonnée en Y \\
    
\end{tabular} \\
\end{tabular}



\functitle{tourelle}

\begin{lst-c++}
struct tourelle {
    position pos;
    int portee;
    int joueur;
    int vie;
    int attaque;
};
\end{lst-c++}

\noindent
\begin{tabular}[t]{@{\extracolsep{0pt}}>{\bfseries}lp{10cm}}
Description~: & Représente une tourelle \\
Champs~: &
\small
\begin{tabular}[t]{@{\extracolsep{0pt}}lp{7cm}}
    
        \textsl{pos}~: & Position de la tourelle \\
    
        \textsl{portee}~: & Portée de la tourelle \\
    
        \textsl{joueur}~: & Joueur qui possède la tourelle \\
    
        \textsl{vie}~: & Nombre de points de vie de la tourelle \\
    
        \textsl{attaque}~: & Nombre de points d'attaque de la tourelle \\
    
\end{tabular} \\
\end{tabular}




\begin{minipage}{\linewidth}
\functitle{info\_case}

\begin{lst-c++}
case_info info_case(position pos)
\end{lst-c++}

\noindent
\begin{tabular}[t]{@{\extracolsep{0pt}}>{\bfseries}lp{10cm}}
Description~: & Retourne le type de la case à l'emplacement `pos` \\


Parametres~: &
\begin{tabular}[t]{@{\extracolsep{0pt}}ll}
    
    
      
        \textsl{pos}~: & Position de la case \\
      
    
  \end{tabular} \\






\end{tabular} \\[0.3cm]
\end{minipage}


\begin{minipage}{\linewidth}
\functitle{tourelles\_joueur}

\begin{lst-c++}
tourelle array tourelles_joueur(int joueur)
\end{lst-c++}

\noindent
\begin{tabular}[t]{@{\extracolsep{0pt}}>{\bfseries}lp{10cm}}
Description~: & Retourne la liste des tourelles qui appartiennent au joueur ``joueur`` \\


Parametres~: &
\begin{tabular}[t]{@{\extracolsep{0pt}}ll}
    
    
      
        \textsl{joueur}~: & Identifiant du joueur \\
      
    
  \end{tabular} \\






\end{tabular} \\[0.3cm]
\end{minipage}


\begin{minipage}{\linewidth}
\functitle{magie}

\begin{lst-c++}
int magie(int joueur)
\end{lst-c++}

\noindent
\begin{tabular}[t]{@{\extracolsep{0pt}}>{\bfseries}lp{10cm}}
Description~: & Retourne la magie que possède le joueur ``joueur`` \\


Parametres~: &
\begin{tabular}[t]{@{\extracolsep{0pt}}ll}
    
    
      
        \textsl{joueur}~: & Numéro du joueur \\
      
    
  \end{tabular} \\






\end{tabular} \\[0.3cm]
\end{minipage}


\begin{minipage}{\linewidth}
\functitle{nb\_sorciers}

\begin{lst-c++}
int nb_sorciers(position pos, int joueur)
\end{lst-c++}

\noindent
\begin{tabular}[t]{@{\extracolsep{0pt}}>{\bfseries}lp{10cm}}
Description~: & Retourne le nombre de sorciers du joueur ``joueur`` sur la case ``pos`` \\


Parametres~: &
\begin{tabular}[t]{@{\extracolsep{0pt}}ll}
    
    
      
        \textsl{pos}~: & Case \\
      
    
      
        \textsl{joueur}~: & Identifiant du joueur \\
      
    
  \end{tabular} \\






\end{tabular} \\[0.3cm]
\end{minipage}


\begin{minipage}{\linewidth}
\functitle{nb\_sorciers\_deplacables}

\begin{lst-c++}
int nb_sorciers_deplacables(position pos, int joueur)
\end{lst-c++}

\noindent
\begin{tabular}[t]{@{\extracolsep{0pt}}>{\bfseries}lp{10cm}}
Description~: & Retourne le nombre de sorciers du joueur ``joueur`` déplaçables sur la case ``pos`` \\


Parametres~: &
\begin{tabular}[t]{@{\extracolsep{0pt}}ll}
    
    
      
        \textsl{pos}~: & Case \\
      
    
      
        \textsl{joueur}~: & Identifiant du joueur \\
      
    
  \end{tabular} \\






\end{tabular} \\[0.3cm]
\end{minipage}


\begin{minipage}{\linewidth}
\functitle{joueur\_case}

\begin{lst-c++}
int joueur_case(position pos)
\end{lst-c++}

\noindent
\begin{tabular}[t]{@{\extracolsep{0pt}}>{\bfseries}lp{10cm}}
Description~: & Retourne le numéro du joueur qui contrôle la case ``pos`` \\


Parametres~: &
\begin{tabular}[t]{@{\extracolsep{0pt}}ll}
    
    
      
        \textsl{pos}~: & Case \\
      
    
  \end{tabular} \\






\end{tabular} \\[0.3cm]
\end{minipage}


\begin{minipage}{\linewidth}
\functitle{tourelle\_case}

\begin{lst-c++}
tourelle tourelle_case(position pos)
\end{lst-c++}

\noindent
\begin{tabular}[t]{@{\extracolsep{0pt}}>{\bfseries}lp{10cm}}
Description~: & Retourne la tourelle située sur la case ``pos`` \\


Parametres~: &
\begin{tabular}[t]{@{\extracolsep{0pt}}ll}
    
    
      
        \textsl{pos}~: & Case de la tourelle \\
      
    
  \end{tabular} \\






\end{tabular} \\[0.3cm]
\end{minipage}


\begin{minipage}{\linewidth}
\functitle{base\_joueur}

\begin{lst-c++}
position base_joueur(int joueur)
\end{lst-c++}

\noindent
\begin{tabular}[t]{@{\extracolsep{0pt}}>{\bfseries}lp{10cm}}
Description~: & Retourne la position de la base du joueur ``joueur`` \\


Parametres~: &
\begin{tabular}[t]{@{\extracolsep{0pt}}ll}
    
    
      
        \textsl{joueur}~: & Identifiant du joueur \\
      
    
  \end{tabular} \\






\end{tabular} \\[0.3cm]
\end{minipage}


\begin{minipage}{\linewidth}
\functitle{constructible}

\begin{lst-c++}
bool constructible(position pos, int joueur)
\end{lst-c++}

\noindent
\begin{tabular}[t]{@{\extracolsep{0pt}}>{\bfseries}lp{10cm}}
Description~: & Retourne vrai si l'on peut construire sur la case ``pos`` \\


Parametres~: &
\begin{tabular}[t]{@{\extracolsep{0pt}}ll}
    
    
      
        \textsl{pos}~: & Case \\
      
    
      
        \textsl{joueur}~: & Identifiant du joueur \\
      
    
  \end{tabular} \\






\end{tabular} \\[0.3cm]
\end{minipage}


\begin{minipage}{\linewidth}
\functitle{chemin}

\begin{lst-c++}
position array chemin(position pos1, position pos2)
\end{lst-c++}

\noindent
\begin{tabular}[t]{@{\extracolsep{0pt}}>{\bfseries}lp{10cm}}
Description~: & Retourne la liste des positions constituant le plus court chemin allant de la case ``pos1`` à la case ``pos2``. Attention : Cette fonction est lente. \\


Parametres~: &
\begin{tabular}[t]{@{\extracolsep{0pt}}ll}
    
    
      
        \textsl{pos1}~: & Case de départ \\
      
    
      
        \textsl{pos2}~: & Case d'arrivée \\
      
    
  \end{tabular} \\






\end{tabular} \\[0.3cm]
\end{minipage}


\begin{minipage}{\linewidth}
\functitle{construire}

\begin{lst-c++}
erreur construire(position pos, int portee)
\end{lst-c++}

\noindent
\begin{tabular}[t]{@{\extracolsep{0pt}}>{\bfseries}lp{10cm}}
Description~: & Construire une tourelle à la position ``pos`` \\


Parametres~: &
\begin{tabular}[t]{@{\extracolsep{0pt}}ll}
    
    
      
        \textsl{pos}~: & Position \\
      
    
      
        \textsl{portee}~: & Portée \\
      
    
  \end{tabular} \\






\end{tabular} \\[0.3cm]
\end{minipage}


\begin{minipage}{\linewidth}
\functitle{supprimer}

\begin{lst-c++}
erreur supprimer(position pos)
\end{lst-c++}

\noindent
\begin{tabular}[t]{@{\extracolsep{0pt}}>{\bfseries}lp{10cm}}
Description~: & Supprimer une tourelle à la position ``pos`` \\


Parametres~: &
\begin{tabular}[t]{@{\extracolsep{0pt}}ll}
    
    
      
        \textsl{pos}~: & Position \\
      
    
  \end{tabular} \\






\end{tabular} \\[0.3cm]
\end{minipage}


\begin{minipage}{\linewidth}
\functitle{tirer}

\begin{lst-c++}
erreur tirer(int pts, position tourelle, position cible)
\end{lst-c++}

\noindent
\begin{tabular}[t]{@{\extracolsep{0pt}}>{\bfseries}lp{10cm}}
Description~: & Tirer avec ``pts`` points de dégât depuis la tourelle sur ``pos`` sur la position ``cible`` \\


Parametres~: &
\begin{tabular}[t]{@{\extracolsep{0pt}}ll}
    
    
      
        \textsl{pts}~: & Nombre de points de dégât \\
      
    
      
        \textsl{pos}~: & Position de la tourelle \\
      
    
      
        \textsl{cible}~: & Position de la cible \\
      
    
  \end{tabular} \\






\end{tabular} \\[0.3cm]
\end{minipage}


\begin{minipage}{\linewidth}
\functitle{creer}

\begin{lst-c++}
erreur creer(int nb)
\end{lst-c++}

\noindent
\begin{tabular}[t]{@{\extracolsep{0pt}}>{\bfseries}lp{10cm}}
Description~: & Créer ``nb`` sorciers dans la base \\


Parametres~: &
\begin{tabular}[t]{@{\extracolsep{0pt}}ll}
    
    
      
        \textsl{nb}~: & Position d'arrivée \\
      
    
  \end{tabular} \\






\end{tabular} \\[0.3cm]
\end{minipage}


\begin{minipage}{\linewidth}
\functitle{deplacer}

\begin{lst-c++}
erreur deplacer(position depart, position arrivee, int nb)
\end{lst-c++}

\noindent
\begin{tabular}[t]{@{\extracolsep{0pt}}>{\bfseries}lp{10cm}}
Description~: & Déplace ``nb`` sorciers de la position ``depart`` jusqu'à la position ``arrivee``. \\


Parametres~: &
\begin{tabular}[t]{@{\extracolsep{0pt}}ll}
    
    
      
        \textsl{depart}~: & Position de départ \\
      
    
      
        \textsl{arrivee}~: & Position d'arrivée \\
      
    
      
        \textsl{nb}~: & Nombre de sorciers à déplacer \\
      
    
  \end{tabular} \\






\end{tabular} \\[0.3cm]
\end{minipage}


\begin{minipage}{\linewidth}
\functitle{assieger}

\begin{lst-c++}
erreur assieger(position pos, position cible, int nb_sorciers)
\end{lst-c++}

\noindent
\begin{tabular}[t]{@{\extracolsep{0pt}}>{\bfseries}lp{10cm}}
Description~: & Attaquer la tourelle à la position ``cible`` depuis la position ``pos`` \\


Parametres~: &
\begin{tabular}[t]{@{\extracolsep{0pt}}ll}
    
    
      
        \textsl{pos}~: & Position des sorciers \\
      
    
      
        \textsl{cible}~: & Position de la tourelle à attaquer \\
      
    
      
        \textsl{nb\_sorciers}~: & Nombre de sorciers attaquant la tourelle \\
      
    
  \end{tabular} \\






\end{tabular} \\[0.3cm]
\end{minipage}


\begin{minipage}{\linewidth}
\functitle{moi}

\begin{lst-c++}
int moi()
\end{lst-c++}

\noindent
\begin{tabular}[t]{@{\extracolsep{0pt}}>{\bfseries}lp{10cm}}
Description~: & Retourne le numéro de votre joueur \\







\end{tabular} \\[0.3cm]
\end{minipage}


\begin{minipage}{\linewidth}
\functitle{adversaires}

\begin{lst-c++}
int array adversaires()
\end{lst-c++}

\noindent
\begin{tabular}[t]{@{\extracolsep{0pt}}>{\bfseries}lp{10cm}}
Description~: & Retourne la liste des numéros de vos adversaires \\







\end{tabular} \\[0.3cm]
\end{minipage}


\begin{minipage}{\linewidth}
\functitle{tour\_actuel}

\begin{lst-c++}
int tour_actuel()
\end{lst-c++}

\noindent
\begin{tabular}[t]{@{\extracolsep{0pt}}>{\bfseries}lp{10cm}}
Description~: & Retourne le numéro du tour actuel \\







\end{tabular} \\[0.3cm]
\end{minipage}


\begin{minipage}{\linewidth}
\functitle{distance}

\begin{lst-c++}
int distance(position depart, position arrivee)
\end{lst-c++}

\noindent
\begin{tabular}[t]{@{\extracolsep{0pt}}>{\bfseries}lp{10cm}}
Description~: & Retourne la distance entre deux positions \\


Parametres~: &
\begin{tabular}[t]{@{\extracolsep{0pt}}ll}
    
    
      
        \textsl{depart}~: & Départ \\
      
    
      
        \textsl{arrivee}~: & Arrivée \\
      
    
  \end{tabular} \\






\end{tabular} \\[0.3cm]
\end{minipage}


\begin{minipage}{\linewidth}
\functitle{annuler}

\begin{lst-c++}
erreur annuler()
\end{lst-c++}

\noindent
\begin{tabular}[t]{@{\extracolsep{0pt}}>{\bfseries}lp{10cm}}
Description~: & Annule la dernière action \\







\end{tabular} \\[0.3cm]
\end{minipage}


